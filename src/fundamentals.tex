\chapter{Theoretical Fundamentals\label{cha:fundamentals}}

\section{A story starts from time evolution quantum mechanics}

Imagine that we are watching a microscopic play, with electrons and atoms as the actors. You blink, and suddenly the stage configuration has shifted. Something has changed. But how do we describe this evolution in physical language?

In classical physics, we start with Newton’s laws to describe motion. Later, Lagrangian and Hamiltonian mechanics provided another description of physics from the perspective of degrees of freedom, symmetries, and conservations. Then came the great 20th-century revolution in physics, which developed the consensus of microscopic scale, and quantum mechanics emerged. In this new framework, that role is played by the Schrödinger equation.

\subsection{Time-dependent Schrödinger equation}

In quantum mechanics, the state of a physical system at any given instant or a wavefunction is represented by a vector $\ket{\psi}$. This state vector encapsulates all the information we can know about the system at time $t$.

But how exactly does this state evolve over time? The evolution is determined by a special operator known as the Hamiltonian $\hat{H}$. Which corresponds physically to the total energy of the conserved systems. This evolution is known as time-dependent Schrödinger equation:

\begin{equation}
    i\hbar\frac{\partial}{\partial t}\ket{\psi(t)}=\hat{H}\ket{\psi(t)}
    \label{tdse}
\end{equation}

Here, the left side describes how quickly and in what manner the quantum state changes over time; And the right side tells us that this rate of change is determined by the Hamiltonian operator, reflecting the total energy configuration of this conserved system. 

If we know the state vector at an initial time $t=0$. This elegant equation provides us with a probability to predict its future states.

\subsection{Time evolution operator}

Suppose that the Hamiltonian is independent of time; for example, the energy configuration does not change explicitly with time. In this specific case, the Schrödinger equation can be solved analytically via a given initial state vector $\ket{\psi(0)}$, we denote a future state vector $\ket{\psi(t)}$ via the expression:

\begin{equation}
    \ket{\psi(t)}=e^{-\frac{i}{\hbar}\hat{H}t}\ket{\psi(0)}
    \label{state_vec}
\end{equation}

Here, we introduce the time evolution operator $\hat{U}(t)$ as:

\begin{equation}
    \hat{U}(t) = e^{-\frac{i}{\hbar}\hat{H}t}
    \label{evolution_ope}
\end{equation}

We can imagine this operator as a predictor, which takes an initial quantum state and moves it forward smoothly and predictably into its future form. A crucial property of this operator is its unitarity:

\begin{equation}
    \hat{U}^\dagger(t)\hat{U}(t)=\hat{U}(t)\hat{U}^\dagger(t) = \hat{I}
    \label{evolution_unitarity}
\end{equation}

\subsection{Generalizing to time-dependent Hamiltonians}

\newpage
\section{Decoding the density matrix}

\newpage
\section{Second quantization for fermions}

\newpage
\section{Exploring the dielectric function and linear optical properties}

\newpage
\section{A brief approach to density functional theory}

We start with the many-body Hamiltonian. For any system, the first term is the Hamiltonian representing the kinetic energy of electrons:

\begin{equation}
    -\frac{\hbar^2}{2m_e}\sum_i\nabla_i^2
    \label{electrons_kinetic}
\end{equation}

The electrons live in the Coulomb potential that is created by the nuclei, there is also Coulomb interaction between electrons and atomic nuclei. We use $\Vec{r}$ to present the position of electrons, $\Vec{R}$ for the position of the nucleus, and $Z$ for the corresponding charge number of the $I$-th atomic nucleus. Then, we get this Coulomb interaction Hamiltonian between electrons and nucleus:

\begin{equation}
    -\sum_{i,I}\frac{Z_I e^2}{|\Vec{r}_i-\Vec{R}_I|}
    \label{electrons_nucleus_Coulomb}
\end{equation}

This term illustrates how the energy of an electron is influenced by the position of each nucleus. Next, we consider the Coulomb interaction between electrons:

\begin{equation}
    \frac{1}{2}\sum_{i\neq j}\frac{e^2}{|\Vec{r}_i-\Vec{r}_j|}
    \label{electrons_Coulomb}
\end{equation}

Now, focusing on the nuclei, we describe their kinetic energy as:   

\begin{equation}
    -\sum_I\frac{\hbar^2}{2M_I}\nabla_I^2
    \label{nucleus_kinetic}
\end{equation}

There is also an interaction between the nuclei themselves:

\begin{equation}
    \frac{1}{2}\sum_{I\neq J}\frac{Z_I Z_J e^2}{|\Vec{R}_I-\Vec{R}_J|}
    \label{nucleus_interaction}
\end{equation}

Summing these terms, we express the Hamiltonian of a many-body system as:

\begin{equation}
    \hat{H}=-\frac{\hbar^2}{2m_e}\sum_i\nabla_i^2
    -\sum_{i,I}\frac{Z_I e^2}{|\Vec{r}_i-\Vec{R}_I|}
    +\frac{1}{2}\sum_{i\neq j}\frac{e^2}{|\Vec{r}_i-\Vec{r}_j|}
    -\sum_I\frac{\hbar^2}{2M_I}\nabla_I^2
    +\frac{1}{2}\sum_{I\neq J}\frac{Z_I Z_J e^2}{|\Vec{R}_I-\Vec{R}_J|}
    \label{full_many_body_hamiltonian}
\end{equation}

In principle, this complex coupled term demands careful attention. Fortunately, due to the much greater mass of the nuclei compared to the electrons, we can approximate by neglecting the nuclei’s kinetic energy term and treating their interaction as constant. This reduces the Hamiltonian to:

\begin{equation}
    \hat{H}=-\frac{\hbar^2}{2m_e}\sum_i\nabla_i^2
    -\sum_{i,I}\frac{Z_I e^2}{|\Vec{r}_i-\Vec{R}_I|}
    +\frac{1}{2}\sum_{i\neq j}\frac{e^2}{|\Vec{r}_i-\Vec{r}_j|}
    +C
    \label{many_body_hamiltonian}
\end{equation}

The Coulomb interaction Hamiltonian term between electrons and nuclei represents the sum of all potentials, giving the Hamiltonian:

\begin{equation}
    \hat{H}=-\frac{\hbar^2}{2m_e}\sum_i\nabla_i^2
    +\sum_i v_\text{ext}(\Vec{r}_i)
    +\frac{1}{2}\sum_{i\neq j}\frac{e^2}{|\Vec{r}_i-\Vec{r}_j|}
    \label{many_body_hamiltonian}
\end{equation}

Considering the parameters in this Hamiltonian, we observe that the many-body Hamiltonian and wave functions for each state $\lambda$ depend on $3 \times N$ spatial coordinates of the electrons and their spin $\sigma$:

\begin{equation}
    \hat{H}(\Vec{r}_1,\Vec{r}_2,\dots,\Vec{r}_N)\Psi(x_1,x_2,\dots,x_N)
    = E_\lambda\Psi_\lambda(x_1,x_2,\dots,x_N)
    \label{many_body_hamiltonian_argument}
\end{equation}

where, $x_i:=(\Vec{r_i},\sigma_i)$.

The main challenge arises from the inseparability of these coordinates. When electron interactions are disregarded, the problem simplifies to 
$N$ independent single-electron problems. If we further overlook the Pauli exclusion principle, the many-body wave function is merely a product of 
$N$ one-body wave functions:

\begin{equation}
    \Psi(x_1,x_2,\dots,x_N)=\psi_{n_1}(x_1)\psi_{n_2}(x_2)\dots\psi_{n_N}(x_N)
    \label{body_function}
\end{equation}

where, $\lambda={n_1,n_2,\dots,n_N}$.

However, to satisfy the Pauli principle—requiring that electrons with identical spins do not occupy the same state—the many-body wave function must be represented as:

\begin{equation}
    \Psi = \frac{1}{(N!)^{1/2}}
    \begin{vmatrix}
    \psi_1(\Vec{r}_1,\sigma_1) & \psi_1(\Vec{r}_2,\sigma_2) & \psi_1(\Vec{r}_3,\sigma_3) & \dots \\
    \psi_2(\Vec{r}_1,\sigma_1) & \psi_2(\Vec{r}_2,\sigma_2) & \psi_2(\Vec{r}_3,\sigma_3) & \dots \\
    \psi_3(\Vec{r}_1,\sigma_1) & \psi_3(\Vec{r}_2,\sigma_2) & \psi_3(\Vec{r}_3,\sigma_3) & \dots \\
    \vdots & \vdots & \ddots
    \end{vmatrix}
\end{equation}

\newpage
\section{An adventure comes from Statistical Physics to Machine Learning (probably)}

